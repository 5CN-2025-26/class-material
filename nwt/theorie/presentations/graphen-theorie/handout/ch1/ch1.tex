Graphentheorie bildet die Grundlage für das Verständnis komplexer 
Vernetzungsstrukturen. Ursprünglich aus mathematischen Fragestellungen 
entwickelt, ist sie heute ein zentrales Werkzeug zur Modellierung und Analyse
moderner Informations- und Kommunikationssysteme. Besonders in der Internet -
und Netzwerktechnologie unterstützt sie die Beschreibung von Topologien, 
die Optimierung von Datenwegen sowie die Bewertung von Zuverlässigkeit und 
Effizienz technischer Infrastrukturen.

\section{Grundlagen}

Ein einfacher Graph $G$ besteht aus Knoten und Kanten. Dabei gilt,

\begin{equation}
    G = (V, E)
\end{equation}

wobei $V$ die endliche Menge an Knoten und $E$ die endliche Menge an Kanten 
darstellt. Genauer lässt sich $E$ als 2-elementige Teilmenge von $V$, also 
$E = [V]^2$, beschreiben. Das bedeudet, dass eine Kante $e$ jeweils durch 
zwei Knoten $v_1$ und $v_2$ definiert wird.
